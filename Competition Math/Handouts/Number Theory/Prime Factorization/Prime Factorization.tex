\documentclass[11pt,paper=letter]{scrartcl}
\usepackage[wide,boxthm]{cjquines}

% socrery https://tex.stackexchange.com/a/69239/122245
\makeatletter
\newdimen\bibindent
\bibindent=16pt
\renewenvironment{thebibliography}[1]
  {\par\footnotesize
   \section*{\refname}
   \@mkboth{\MakeUppercase{\refname}}{\MakeUppercase{\refname}}
   \list{\@biblabel{\arabic{enumi}}}%
        {\settowidth\labelwidth{\@biblabel{#1}}%
         \leftmargin\labelwidth
         \advance\leftmargin\labelsep
         \advance\leftmargin\bibindent
         \itemindent -\bibindent
         \listparindent \itemindent
         \parsep \z@
         \usecounter{enumi}%
         \let\p@enumi\@empty
         \renewcommand\theenumi{\arabic{enumi}}}%
     \renewcommand\newblock{\hskip .11em \@plus.33em \@minus.07em}%
     \sloppy\clubpenalty4000\widowpenalty4000%
     \frenchspacing\footnotesize}
     {\endlist}
\makeatother

\let\faBoltOld\faBolt
\renewcommand{\faBolt}{{\relsize{-1}\faBoltOld}}

\begin{document}

\title{Prime Factorization}
\author{Chirag Samantaroy}
\date{\today}

\maketitle

\section{Definitions}
\begin{itemize}
    \item We say that $a$ divides $b,$ denoted as $a \mid b,$ if $b = ac$ for some integer $c.$
    \item The prime factorization of a number $n$ is the representation of $n$ as a product of not necessarily distinct primes, written as $$n=p_1^{e_1}p_2^{e_2}\cdots p_k^{e_k}$$ where $p_i$ is the $i$th prime and $e_i$ is the power of $p_i$ in $n.$ 
\end{itemize}

\begin{probboxed}[Fundamental Theorem of Arithmetic]
Every integer has a unique prime factorization up to the ordering of the prime powers.
\end{probboxed}

\section{GCD/LCM}
The \bluebf{greatest common divisor} of two or more integers, which are not all zero, is the largest positive integer that divides each of the integers.

Two integers are \bluebf{relatively prime} if there is no integer greater than one that divides them both (i.e. $\gcd(m, n) = 1 \iff m, n$ are relatively prime).

Suppose $m, n$ are relative prime integers. By the definition of relatively prime, no prime number can divide both $m, n,$ so if the prime factorizations are \[\prod_{i=1}^{\infty} p_i^{m_i} \text{ and } \prod_{i=1}^{\infty} p_i^{n_i}\] respectively, then one of $m_i, n_i$ is $0$ for each $i.$ Actually, it turns out that this is sufficient, since \[\gcd(m, n) = \prod_{i=1}^{\infty} p_i^{\min(m_i, n_i)} = \prod_{i=1}^{\infty} p_i^0 = 1.\]

The \bluebf{least common multiple} of two or more integers, which are not all zero, is the smallest positive integer that is a multiple of each of the integers.

Going back, again suppose the prime factorizations of two integers $m$ and $n$ \[\prod_{i=1}^{\infty} p_i^{m_i} \text{ and } \prod_{i=1}^{\infty} p_i^{n_i}\] respectively. Then, \[\lcm(m, n) = \prod_{i = 1}^{\infty} p_i^{\max(m_i, n_i)}.\] Also, it's well-known the formula: \[\lcm(m, n) = \frac{|mn|}{\gcd(m, n)}.\]

\section{Number and Sum of Divisors}

\begin{probboxed}[Number of Divisors]
The \bluebf{number of divisors} of $n=p_1^{e_1}p_2^{e_2}\cdots p_k^{e_k}$ is \[\tau(n) = \prod_{i = 1}^{k} e_i + 1.\]
\end{probboxed}
Any divisor of $n = p_1^{e_1}p_2^{e_2}\cdots p_k^{e_k}$ is of the form $p_1^{a_1}p_2^{a_2}\cdots p_k^{e_k}$ for some $a_1 \in \{0, 1, \dots, e_1 - 1, e_1\}, a_2 \in \{0, 1, \dots, e_2 - 1, e_2\},$ and so on and so forth till $a_k \in \{0, 1, \dots, e_k - 1, e_k\}.$ (Why?) Hence, there are $e_1 + 1$ choices for $a_1, e_2 + 1$ choices for $a_2,$ and so on and so forth, for a total of \[(e_1 + 1)(e_2 + 1)\cdots(e_k + 1) = \prod_{i = 1}^{k} e_i + 1\]  divisors.

\begin{probboxed}[Sum of Divisors]
The \bluebf{sum of divisors} of $n=p_1^{e_1}p_2^{e_2}\cdots p_k^{e_k}$ is \[\sigma(n) = \prod_{j = 1}^{k}\sum_{i = 0}^{e_j} p^i.\]
\end{probboxed}

\section{Examples}
\begin{exboxed}[PuMAC 2007/Number Theory B1]
If you multiply all positive integer factors of $24$, you get $24^x$. Find $x$.
\end{exboxed}
The prime factorization of $24$ is $2^3 \cdot 3,$ so there are a total of $(3 + 1)(1 + 1) = 4 \cdot 2 = 8$ positive integer factors, and therefore four pairs of factors that multiply to $24$. So, our answer is $24^4 \implies \boxed{4}.$
\begin{exboxed}[PuMAC 2011/Number Theory A1/B3]
The only prime factors of an integer $n$ are 2 and 3. If the sum of the divisors of $n$ (including itself) is $1815$, find $n$.
\end{exboxed}
Let $n=2^a3^b$. We have \[\left(1+2+\dots+2^a\right)\left(1+3+\dots+3^b\right)=1815.\] By the finite geometric series formula, 
\begin{align*}
\left(\frac{1 - 2^{a + 1}}{1 - 2}\right)\left(\frac{1 - 3^{b + 1}}{1 - 3}\right) = 1815\\
\implies \left(\frac{1 - 2^{a + 1}}{-1}\right)\left(\frac{1 - 3^{b + 1}}{-2}\right) = 1815\\
\implies \frac{(1 - 2^{a + 1})(1 - 3^{b + 1})}{2} = 1815\\
\implies \left(1 - 2^{a + 1}\right)\left(1 - 3^{b + 1}\right) = 3630\\
\implies (2^{a + 1} - 1)(3^{b + 1} - 1) = 3630.
\end{align*}
Notice that $3630 = 15 \cdot 242 = (2^4 - 1)(3^5 - 1).$ So, 
\begin{align*}
    a + 1 = 4 \implies a = 3\\
    b + 1 = 5 \implies b = 4.
\end{align*}
And therefore, our answer is $n = 2^33^4 = \boxed{648}.$
\begin{exboxed}[AMC 10B 2013/9]
Three positive integers are each greater than $1$, have a product of $ 27000 $, and are pairwise relatively prime. What is their sum?
\end{exboxed}
The prime factorization of $27000$ is $2^3 \cdot 3^3 \cdot 5^3.$ Since $2^3 = 8, 3^3 = 27, 5^3 = 125$ are each greater than $1,$ have a product of $2700,$ and are pairwise relatively prime, we say these are the aforementioned three positive integers in the problem statement. Thus, their sum gives an answer of $8 + 27 + 125 = \boxed{160}.$
\begin{exboxed}[AMC 10B 2013/24]
A positive integer $n$ is \emph{nice} if there is a positive integer $m$ with exactly four positive divisors (including $1$ and $m$) such that the sum of the four divisors is equal to $n$. How many numbers in the set $\{2010, 2011, 2012,\ldots,2019\}$ are nice?
\end{exboxed}
For an integer $n$ to have four positive divisors, it must either be a perfect cube or a product of two primes. None of the numbers in the set $\{2010, 2011, 2012, \ldots, 2019\}$ are perfect cubes (why?), so we can ignore that case. 

If $n = pq$ for two primes $p$ and $q,$ then the sum of the divisors of $n$ is \[(1 + p)(1 + q),\] which is equal to $n$ (assuming $n$ is nice).

\bluebf{Case 1: Either $p$ or $q$ are $2.$} Without loss of generality, assume $p = 2.$ Then, $p + 1 = 3 \mid n,$ and $q + 1$ remains even. Thus, the only numbers that work are even multiples of $3$ which are $2010$ and $2016.$ However, to check these are nice, we need to check if $\frac{2010}{3} - 1$ and $\frac{2016}{3} - 1$ are prime, which they are not. So, we see that in this case none of them work.

\bluebf{Case 2: Both $p$ and $q$ are odd primes.} This means that, $p + 1$ and $q + 1$ are both even, resulting in $4 \mid n.$ Thus, the only numbers that work are multiples of $4$ which are $2012$ and $2016.$ Notice that, $2012 = 4 \cdot 503,$ and $503 - 1$ is not a prime. This leaves $2016,$ which we can show works.

Thus, there is only $\boxed{1}$ number in the set $\{2010, 2011, 2012,\ldots,2019\}$ that is nice.

\begin{exboxed}[AMC 10B 2016/12]
Two different numbers are selected at random from $\{1, 2, 3, 4, 5\}$ and multiplied together. What is the probability that the product is even?
\end{exboxed}
We complementary count. The only possible way to obtain a product that is odd is if the two numbers selected are both odd. The probability of this happening is $\frac{\binom{3}{2}}{\binom{5}{2}} = \frac{3}{10}.$ Thus, the probability that the product is even is $1 - \frac{3}{10} = \frac{7}{10}.$

\begin{exboxed}[AIME II 2019/3]
Find the number of $7$-tuples of positive integers $(a,b,c,d,e,f,g)$ that satisfy the following systems of equations:
\begin{align*}
abc&=70,\\
cde&=71,\\
efg&=72.
\end{align*}
\end{exboxed}
From the system of equations, we deduce 
\begin{align*}
    c \mid 70 \text{ and } c \mid 71 \implies c \mid \gcd(70, 71) = 1.\\
    e \mid 71 \text{ and } e \mid 72 \implies e \mid \gcd(71, 72) = 1.
\end{align*}
Thus, $c = e = 1.$ Substituting this into the system of equations 
\begin{align*}
    ab = 70,\\
    d = 71,\\
    fg = 72.
\end{align*}
There are $8$ factors of $70 = 2 \cdot 5 \cdot 7,$ giving $4$ pairs of $(a, b),$ and $12$ factors of $72 = 2^3 \cdot 3^2,$ giving $6$ pairs of $(f, g).$ However, remember $a$ and $b$ can be interchanged, so this gives $4 \cdot 2 = 8$ choices for $(a, b)$ and $6 \cdot 2 = 12$ choices for $(f, g).$ This gives us an answer of $\boxed{96}.$

\begin{exboxed}[AIME I 2020/4]
Let $S$ be the set of positive integers $N$ with the property that the last four digits of $N$ are $2020,$ and when the last four digits are removed, the result is a divisor of $N.$ For example, $42,020$ is in $S$ because $4$ is a divisor of $42,020.$ Find the sum of all the digits of all the numbers in $S.$ For example, the number $42,020$ contributes $4+2+0+2+0=8$ to this total.
\end{exboxed}
There is a positive integer $m$ such that $N = 10^4m + 2020.$ $N$ has the property that $m \mid N,$ and since $m \mid 10^4m,$ we need $m \mid 2020.$ 

\begin{exercise}
    Finish the problem off from here!
\end{exercise}
    

\begin{exboxed}[AoPS Wiki]
Find the largest integer $k$ for which $2^k$ divides $27!$
\end{exboxed}
For a problem such as this, we need to refer to Legendre's Formula:

\begin{probboxed}[Legendre's Formula]
\bluebf{Legendre's Formula} states that \[v_p(n!)=\sum_{i=1}^{\infty} \left\lfloor \dfrac{n}{p^i}\right\rfloor,\] where $p$ is a prime and $v_p(n!)$ is the exponent of $p$ in the prime factorization of $n!.$
\end{probboxed}
Using Legendre's Formula, substituting $n=27$ and $p=2$ gives
\begin{align*}v_2(27!)=&\left\lfloor\frac{27}{2}\right\rfloor+\left\lfloor\frac{27}{2^2}\right\rfloor+\left\lfloor\frac {27}{2^3}\right\rfloor+\left\lfloor\frac{27}{2^4}\right\rfloor\\
=& 13+6+3+1\\
=& 23\end{align*}
which means that the largest integer $k$ for which $2^k$ divides $27!$ is $\boxed{23}$.
\end{document}
