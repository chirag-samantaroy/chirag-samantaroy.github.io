\documentclass{beamer}
\usepackage[utf8]{inputenc}
\usepackage{hyperref}
\usetheme{Madrid}
\usecolortheme{seahorse}

\title{The Central Dogma of Molecular Biology}
\author{Chirag Samantaroy}
\date{\today}

\begin{document}

\maketitle

\begin{frame}
\frametitle{Introduction}
The Central Dogma of Molecular Biology is DNA $\rightarrow$ RNA $\rightarrow$ Proteins.

\bigskip \noindent To explain further, it's essentially stating
\begin{enumerate}
    \item RNA copies the instructions to make a protein from DNA.
    \item RNA uses those instructions to make a protein.
\end{enumerate}
in chronological order.
\end{frame}

\begin{frame}{Comparing and Contrasting DNA and RNA}
\begin{center}
\begin{tabular}{ c|c }
DNA & RNA\\ \hline
Deoxyribose & Ribose \\ \hline
2 Strands & 1 Strand \\ \hline
A, T, C, G & A, U, C, G
\end{tabular}
\end{center}
\end{frame}

\begin{frame}{Nitrogenous Base Pairs of DNA and RNA}
\begin{center}
\begin{tabular}{ c|c }
DNA & RNA \\ \hline
Adenine & Adenine \\ \hline
Thymine & Uracil \\ \hline
Cytosine & Cytosine \\ \hline
Guanine & Guanine
\end{tabular}  
\end{center}
In DNA, adenine always pairs with thymine, and cytosine always pairs with guanine.

\bigskip In RNA, adenine always pairs with uracil, and cytosine always pairs with guanine.

\end{frame}

\begin{frame}{Transcription}
\begin{alertblock}{Definition (Transcription)}
``In \textbf{transcription}, the DNA sequence of a gene is transcribed (copied out) to make an RNA molecule." - \href{https://www.khanacademy.org/science/ap-biology/gene-expression-and-regulation/transcription-and-rna-processing/a/overview-of-transcription}{Khan Academy}
\end{alertblock}

\begin{example}
Find the corresponding mRNA molecule to the following DNA sequence of a gene:
\begin{center}
    ATGCCGCTATC
\end{center}
\end{example}
\begin{enumerate}
    \item<1-> A(denine) pairs with U(racil)
    \item<2-> T(hymine) pairs with A(denine)
    \item<3-> G(uanine) pairs with C(ytosine) and vice versa
    \item<4-> Utilizing this information, the corresponding mRNA molecule would  be
    \begin{center}
        UACGGCGAUAG
    \end{center}
\end{enumerate}

\end{frame}

\begin{frame}{RNA Splicing}
After transcription, two nucleotide sequences are present in the mRNA molecule: Introns and Exons.

\bigskip Now, introns are irrelevant to the development of proteins, whereas the exons contribute to the development of proteins.

\bigskip Thus, we want to get rid of the introns and we want to keep the exons. We do this by RNA Splicing.
\end{frame}

\begin{frame}{Codons}
\begin{alertblock}{Definition (Codon)}
``A \textbf{codon} is a trinucleotide sequence of DNA or RNA that corresponds to a specific amino acid." - National Human Genome Research Institute
\end{alertblock}
Consider the following mRNA molecule:
\begin{center}
    AUGACGGUUUGA
\end{center}
Our codons are AUG, ACG, GUU, and UGA. 
\begin{enumerate}
    \item AUG -- Start Codon (Codes for Methionine)
    \item ACG -- Codes for Threonine 
    \item GUU -- Codes for Valine
    \item UGA -- Stop/Termination Codon
\end{enumerate}
AUG is \textit{always} the start codon.

There are three stop/termination codons: UGA, UAG, UAA.
\end{frame}

\begin{frame}{tRNA}

\end{frame}

\begin{frame}{Developing Proteins}
The genetic code contains the set of instructions to develop the protein, with the exons contributing. 

\bigskip Once the protein is fully developed, this completes the Central Dogma of Molecular Biology Process, and it repeats further for other DNA sequences.
\end{frame}

\end{document}
