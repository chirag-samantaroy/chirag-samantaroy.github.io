\documentclass[12pt]{scrartcl}
\usepackage[sexy]{comet}
\usepackage[legalpaper, portrait, margin=0.5in]{geometry}
\usepackage{graphicx}
\usepackage{float}

\title{Simple Machines and Mechanical Advantage}
\author{Chirag Samantaroy}
\date{\today}

\setlength{\marginparwidth}{2cm}

\begin{document}

\maketitle

\tableofcontents

\pagebreak

\section{Definitions}
\begin{defn}[Energy]
\textbf{Energy} is the ability to do work. 
\end{defn}
\begin{defn}[Work]
\textbf{Work} is a force applied over a distance; work equals force times distance.
\end{defn}
\begin{defn}[Force]
\textbf{Force} is a push or a pull.
\end{defn}
\section{Why do we use simple machines?}
We use simple machines because they make work easier by reducing force or changing the direction of force.
\section{Work input and output}
\begin{itemize}
    \item Work input is the amount of work done on a machine.
    \begin{itemize}
        \item The input force times the input distance
    \end{itemize}
    \item Work output is the amount of work done by a machine.
    \begin{itemize}
        \item The output force times the output distance
    \end{itemize}
\end{itemize}
For an ideal machine, the work input and work output are always the same.
\section{Inclined Plane}
\begin{itemize}
    \item An inclined plane is a flat surface that is higher on one end.
    \item Inclined planes make the work of moving things easier.
\end{itemize}
\subsection{Inclined Plane - Mechanical Advantage}
\begin{itemize}
    \item The mechanical advantage of an inclined plane is equal to the length of the slope divided by the height of the inclined plane.
    \item While the inclined plane produces a mechanical advantage, it does so by increasing the distance through which the force must move.
\end{itemize}
\section{Screw}
The screw is an inclined plane wound around a central cylinder.

\bigskip \noindent The mechanical advantage of a screw can be calculated by dividing the circumference by the pitch ($1$ divided by the number of turns per inch) of the screw.
\section{Wedges}
\begin{itemize}
    \item Two inclined planes joined back to back.
    \item Wedges are used to split things.
\end{itemize}
\subsection{Wedge - Mechanical Advantage}
\begin{itemize}
    \item The mechanical advantage of a wedge can be found by dividing the length or either slope (S) by the thickness (T) of the big end.
    \item As with the inclined plane, the mechanical advantage gained by using a wedge requires a corresponding increase in distance. 
\end{itemize}
\section{Lever}
\subsection{First Class Lever}
Fulcrum is between EF (effort) and RF (load). \textbf{Effort moves farther than Resistance.} Multiplies EF and changes its direction.

\bigskip \noindent The mechanical advantage of a lever is the ratio of the length of the lever on the applied force side of the fulcrum to the length of the lever on the resistance force side of the fulcrum.

\begin{figure}[htp]
    \centering
    \includegraphics[height=50mm]{image.png}
    \caption{First Class Lever}
\end{figure}

\subsection{Second Class Lever} 
RF (load) is between fulcrum and EF (effort). \textbf{Effort moves farther than Resistance.} Multiplies EF, but does not change its direction.

\bigskip \noindent The mechanical advantage of a lever is the ratio of the distance from the applied force to the fulcrum to the distance from the resistance force to the fulcrum.

\begin{figure}[htp]
    \centering
    \includegraphics[height=50mm]{image19.png}
    \caption{Second Class Lever}
\end{figure}

\subsection{Third Class Lever}
EF (effort) is between fulcrum and RF (load). Does not multiply force. \textbf{Resistance moves farther than Effort.} Multiplies the distance the effort force travels.

\bigskip \noindent The mechanical advantage of a lever is the ratio of the distance from the applied force to the fulcrum to the distance of the resistance force to the fulcrum.

\begin{figure}[htp]
    \centering
    \includegraphics[height=50mm]{image21.png}
    \caption{Third Class Lever}
\end{figure}

\section{Pulleys}
\begin{itemize}
    \item Pulley are wheels and axles with a groove around the outside.
    \item A pulley needs a rope, chain or belt around the groove to make it do work.
\end{itemize}
\subsection{Fixed Pulley}
A fixed pulley changes the direction of a force; however, it does not create a mechanical advantage. 
\begin{figure}[htp]
    \centering
    \includegraphics[height=50mm]{image27.png}
    \caption{Fixed Pulley}
\end{figure}
\subsection{Movable Pulley}
The mechanical advantage of a moveable pulley is equal to the number of ropes that support the moveable pulley. 
\begin{figure}[H]
    \centering
    \includegraphics[height=50mm]{image28.png}
    \caption{Movable Pulley}
\end{figure}
\subsection{Combined Pulley}
\begin{itemize}
    \item The effort needed to lift the load is less than half the weight of the load.
    \item \textbf{The main disadvantage is it travels a very long distance.}
\end{itemize}
\begin{figure}[H]
    \centering
    \includegraphics[height=50mm]{image30.png}
    \caption{Combined Pulley}
\end{figure}
\section{Wheel and Axle}
\begin{itemize}
    \item The axle is stuck rigidly to a large wheel. Fan blades are attached to the wheel. When the axle turns, the fan blades spin. 
\end{itemize}
\begin{figure}[H]
    \centering
    \includegraphics[height=50mm]{image31.png}
    \caption{Wheel and Axle}
\end{figure}
\begin{itemize}
    \item The mechanical advantage of a wheel and axle is the ratio of the radius of the wheel to the radius of the axle.
    \item The wheel and axle can also increase speed by applying the input force to the axle rather than a wheel. This increase is computed like mechanical advantage.
\end{itemize}
\subsection{Gears}
\begin{itemize}
    \item Each gear in a series reverses the direction of rotation of the previous gear. The smaller gear will always turn faster than the larger gear. 
\end{itemize}
\begin{figure}
    \centering
    \includegraphics[height=50mm]{image34.png}
    \caption{Gears}
\end{figure}
\section{Mechanical Advantage}
\subsection{What is Mechanical Advantage?}
Mechanical advantage is how many time the simple machine increases your force, distance or speed. It is a ratio so there are no units. If it is greater than 1, it multiplies force, but if it is less than one, it multiplies speed or distance. 

\subsection{Ideal v.s. Real Mechanical Advantage}
\begin{itemize}
    \item \textbf{Ideal:} It doesn't take friction into account; measured with lengths, gear teeth, and number of rope segments.
    \item \textbf{Real:} It takes friction into account; only measured in force(s). 
\end{itemize}
\noindent Real mechanical advantage \textbf{always} equals the output (resistance) force divided by the input (effort) force.

\bigskip \noindent Efficiency is the real mechanical advantage divided by the ideal mechanical advantage.

\subsection{Inclined Plane}
Ideal mechanical advantage equals the slope height divided by the height.

\subsection{First, Second, and Third Class Lever}
Ideal mechanical advantage equals the effort arm length divided by the resistance (load) arm length.
\begin{itemize}
    \item The effort arm length is the length from the fulcrum to the center of the effort force.
    \item The resistance arm length is the length from the fulcrum to the center of the resistance force.
\end{itemize}
\subsection{Pulley}
Ideal mechanical advantage equals the number of rope segments supporting the weight.
\end{document}

